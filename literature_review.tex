\chapter{Literature Review} \label{literature_review}
Over last two decades, a lot of methods have been studied to develop a robust gait recognition system~\cite{Rida_19}. However, robust recognition is still challenging due to the presence of large intra-class variations in person's gait which substantially changed the performance of recognition. In this section, we briefly discuss the existing literature proposed for robust gait recognition which are closely related to our work. 

%-------------------------------------------------------------------------
\section{Gait recognition}
Gait Recognition techniques can generally be classified into two categories: 1) appearance-based methods, and 2) model-based ones. The former is based on the traditional way of extracting appearance features from human silhouettes. Most of the previous work following this approach~\cite{Han_06, Bashir_09, Lam_11} used silhouette masks as the main source of information and extracted features that show how these mask change. The most popular descriptor of gait used in such work is gait energy image (GEI)~\cite{Han_06}, a binary mask averaged over the gait cycle of human figure. Though, there are many other alternatives for GEI, e.g. gait entropy image (GEnI)~\cite{Bashir_09}, and gait flow images~\cite{Lam_11} due to it's insensitiveness of incidental silhouettes error, GEI is considered as the most stable and effective gait descriptors.  

The second one is, a model-based approach~\cite{Ariyanto_11, Tafazzoli_10, Feng_16}, which is based on modeling of human body structure and local movement patterns of different body parts. Model-based approaches are generally invariant to various intra-class variations like view angle, clothing, and carrying bag which can greatly affect the appearance of the subject.  However, it mainly relies on extraction of body parameters like stride length, height, knee, torso, and hip from raw video stream which are computational intensive.


%-------------------------------------------------------------------------
\section{Deep learning for gait recognition}
In recent years due to the powerful feature learning abilities of deep neural networks, deep learning-based methods have significantly improved the performance of gait recognition. Convolutional neural network (CNN) based gait recognition methods~\cite{Wu_17, Shiraga_16, Wolf_16,Castro_17} has gained increasing popularity among other deep learning-based methods due to its ability to automatically learn gait features from the given training images. Wu \textit{et al.}~\cite{Wu} performed cross-view gait recognition by developing three convolutional layer network using subject's GEI as input. Shiraga \textit{et al.}~\cite{Shiraga_16} designed a four-layer CNN network consisting two convolutional layers and two fully connected layers for large-scale gait recognition on OU-ISIR database. In~\cite{Wolf_16}, Wolf \textit{et al.} used 3D convolutions for multi-view gait recognition  by capturing spatio-temporal features from raw images and optical flow information. In addition, several other methods were proposed where CNN was combined to traditional machine learning algorithm such as PCA, SVM, and Bayesian classifier. For example, In~\cite{Castro_17} CNN was first used as a feature extractor, thereafter SVM was used to classify those features. In~\cite{Yu02}, Yu \textit{et al.} used generative adversarial nets to design a feature extractor in order to learn the invariant features. 


%-------------------------------------------------------------------------
\section{Pose-Based gait recognition}
Recently, deep learning-based methods have also been used in real-time 2D pose estimation from image and video. The task of pose estimation involves estimating the locations of body parts. It can broadly be classified into two categories single-person and multi-person pose estimation. To recognize multi-person pose, Cao \textit{et al.}~\cite{Cao} developed a deep neural network based regression to estimate the coordinates of body parts. In this work, we used their pretrained model to get an accurate 2D human pose estimation on our experimental dataset.

With the advent of pose-estimation algorithm in computer vision, the recognition of human gait based on pose information has received much more attention~\cite{Feng,Liao} due to its effective representation on gait dynamics and robustness towards covariates condition variations. Feng et al.~\cite{Feng} used the human body joint heatmap to describe each frame. They fed the joint heatmap of consecutive frames to long short term memory (LSTM). Their gait features are the hidden activation values of the last timestep. In~\cite{Liao}, Liao \textit{et al.} construct a temporal-spatial network (PTSN) to extract gait features from 2D human pose information.

In recent years, a large number of methods in computer vision have been proposed to employ recurrent neural network due to its ability to model the long-term contextual information of temporal sequences effectively in the complicated activity recognition task~\cite{Song,Du}. Song \textit{et al.}~\cite{Song} proposed an end-to-end spatial and temporal attention model with LSTM for human action recognition from skeleton data. In~\cite{Du}, Du \textit{et al.} proposed an end-to-end hierarchical RNN network for skeleton based action recognition. They divided the human skeleton into five different parts and then separately feed them into five subnetworks. Our approach in gait recognition is similar to these approaches of activity recognition. In our work, we design a novel RNN architecture to effectively model the temporal dynamics of human 2D pose sequence.
