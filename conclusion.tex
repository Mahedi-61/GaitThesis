\chapter{Conclusion}\label{conclusion}


\section{Summary of Our Work}
The main task of our thesis is to develop an effective algorithm for the Planted Motif Search problem. Our proposed algorithm qPMS-Sigma tries to find the $ (l, d) $-motifs for $ n $ given sequences. It is an exact version of the PMS algorithm. qPMS-Sigma is based on the previous PMS algorithms qPMSPrune, qPMS7, TraverStringRef and PMS8. In our proposed algorithm we introduce clever techniques to compress the input sequences and thus space complexity is improved. We also include a faster comparison technique of the $ l $-mers by adding bitwise comparison techniques. As we know the PMS is a NP-Hard problem, it is exponential in terms of time. So, parallel implementation techniques are proposed at the end of our work. 


\section{Future Prospects of Our Work}
Though in our thesis we have given some ideas about parallel implementation of our algorithm, it is not implemented and tested on multiprocessor system. All comparison
of our algorithm with the stated algorithms are done in single processor system. To understand the real speedup and slackness we need to experiment in a system involving a network of nodes having multiple processors. Our work is the extension of PMS8 codes which involves openMPI libraries. In future we want to continue our work using OpenMPI project which is a open source Message Passing Interface. Apart from the parallelism, more advanced pruning conditions can be used to reduce the search space. Different randomized approaches can also be included for improving the runtime, though it will make the exact version of our algorithm approximate. 




\endinput