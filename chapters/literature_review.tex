\chapter{Literature Review} \label{ch:literature_review}
Over the last two decades, several methods have been studied to develop a robust gait recognition system~\cite{Rida_19}. However, robust recognition is still challenging due to the presence of large intraclass variations in a person's gait which substantially changed the performance. In this chapter, we briefly discuss the literature of the two categories of existing gait recognition techniques: appearance-based and model-based methods.
 Next, we review some of the recent deep learning-based gait recognition approaches which are closely related to our work. 


%-------------------------------------------------------------------------
\section{Appearance-based Methods} \label{sec:appearance_based_methods}
Most of the previous work following this approach~\cite{Han_06, Bashir_09, Lam_11} used human silhouette masks as the main source of information and extracted features that show how these mask change. The most popular gait representation employed in such work is gait energy image (GEI)~\cite{Han_06}, a binary mask computed through aligning and averaging the silhouettes over the complete gait cycle. Though there are many other alternatives for GEI, e.g., gait entropy image (GEnI)~\cite{Bashir_09}, and gait flow image (GFI)~\cite{Lam_11}, due to its in-sensitiveness of incidental silhouettes error, it has been considered as the most stable gait features.  It can achieve good performance under controlled and cooperative environments, but does not show robustness when the view angle and clothing condition change. 

In order to reduce drastic change of the shape of GEI, Huang \textit{et al.}~\cite{Huang_12} fused two new gait representation: shifted energy image and the gait structural profile to increase the robustness to some classes of structural variations. But, the performance of this method is not good enough due to the loss of temporal information while calculating GEI. In~\cite{Chao_19}, GaitSet has been proposed where a gait is regarded as a set consisting of independent frames rather than a template or sequence. Though it handled cross-view conditions very well, it is not good enough in handling cross-carrying and cross-clothing conditions. 

These appearance-based methods in gait recognition are sensitive to the covariate factors since the extraction of human silhouettes is affected by the changes in lighting. Moreover, when the shape of the human body and appearance change substantially, the performance of appearance-based methods severely degrades. Therefore, these methods are not completely robust toward these covariate change.



%-------------------------------------------------------------------------
\section{Model-based Methods} \label{sec:model_based_methods}
In contrast, model-based~\cite{Yam_04, Ariyanto_11, Tafazzoli_10, Feng_16} gait recognition exploits features based on the shape of human body parts and the dynamics of the motion of each of these parts. The salient advantage of the model-based approach is that, as  opposed to silhouette-based approaches, it can efficiently handle many covariate changes such as view angle, body appearance and shape, so, these methods show robustness toward these variations. 

These methods are based on the extraction and modeling of the human body structure as well as the local movement pattern of these parts. Therefore, this approach is often built with a structural and a motion model to capture both static as well as dynamic information of gait. For example, In~\cite{Yam_04}, Yam \textit{et al.} developed an automated model-based approach to recognize people using walking as well as running gait by analyzing the leg motion. They used the Biomechanics of human locomotion and coupled oscillators and employed a bilateral symmetric and an analytical model to successfully extract the leg motion. Ariyanto \textit{et al.}~\cite{Ariyanto_11} employed a structural model including articulated cylinders for fitting the 3D volumetric subject data at each joint to model the lower legs. In~\cite{Tafazzoli_10}, authors presented a model-based approach where they captured the discriminatory features of gait by analyzing the leg and arm movements. For recognition, they used K-nearest neighbor classifier and Fourier components of the joint angle. 

So, Model-based approaches are generally invariant to various intraclass variations like clothing, carrying and view angle variations, etc. However, the main drawback of this approach is the extraction process of body parameters like height, knee, and torso which is computationally expensive and highly dependent on the quality of the video.


%-------------------------------------------------------------------------
\section{Deep Learning for Gait Recognition} \label{sec:deep_learning_gait_rec}
Due to its powerful feature learning abilities, convolutional neural networks (CNNs) have achieved great success in object recognition task in recent years. Several CNN-based gait recognition methods~\cite{Wu_17, Shiraga_16, Wolf_16, Zhang_16, Yu_17, Yu_19} have been proposed which can automatically learn robust gait features from the given training samples. Additionally, using CNNs, we now can execute feature extraction and perform recognition within a single framework using train samples. Wu \textit{et al.}~\cite{Wu_17} performed cross-view gait recognition by developing three convolutional layer network using the subject's GEI as input. Shiraga \textit{et al.}~\cite{Shiraga_16} designed a eight-layered CNN network, GEINet, which consist of two sequential triplets of convolution, pooling, normalization layers, and two fully connected layers for large-scale gait recognition on OU-ISIR database. 

In~\cite{Wolf_16}, Wolf \textit{et al.} used 3D convolutions for multi-view gait recognition by capturing spatio-temporal features from raw images and optical flow information. A Siamese neural network-based gait recognition system has been developed in~\cite{Zhang_16} where GEI was feed as input. In~\cite{Yu_17}, Yu \textit{et al.} used generative adversarial nets to design a feature extractor in order to learn the invariant features. In~\cite{Yu_19}, they further improved the GAN-based method by adopting a multi-loss strategy to optimize the network to increase the inter-class distance and to reduce the intraclass distance at the same time.


%-------------------------------------------------------------------------
\section{Pose Estimation} \label{sec:pose_estimation}
In recent years, there has been a huge interest in the study of deep learning-based approaches for the task of real-time pose estimation from image and video. The task of pose estimation mainly involves localizing the keypoints of human figure to estimate the locations of different body parts~\cite{Wei_16, Cao_19}.  

Authors in~\cite{Wei_16} introduced Convolutional Pose Machines (CPMs) for the task of articulated pose estimation. It consists of a sequence of convolutional networks that repeatedly produce 2D belief maps for the location to make a dense predictions at each image location. CPMs are completely differentiable and their multi-stage architecture can be trained end to end. 

To recognize multi-person pose, Cao \textit{et al.}~\cite{Cao_19} developed a deep CNN-based regression method to estimate the association between anatomical parts in the image. Their bottom-up method achieved state-of-the-art performance on multiple benchmark datasets. 

In this work, we employed their pretrained model to get an accurate 2D pose estimation on our experimental dataset.



%-------------------------------------------------------------------------
\section{Pose-based Gait Recognition} \label{sec:pose_based_gait_rec}
With the advent of the pose-estimation algorithms in computer vision, the recognition of human gait based on pose information has received much more attention~\cite{Feng_16, Liao_17, Liao_19} due to its effective representation of gait features and robustness toward covariate condition variations. Feng \textit{et al.}~\cite{Feng_16} used the human body joint heatmap to describe each frame. They fed the joint heatmap of consecutive frames to long short term memory (LSTM). Their gait features are the hidden activation values of the last timestep. In~\cite{Liao_17}, Liao et al. constructed a temporal-spatial network (PTSN) to extract the spatial-temporal features of gait from 2D human pose information. Authors in~\cite{Liao_19}, employed 3D pose estimation in their PoseGait network to extract the spatial-temporal gait features and achieved better performance compared with 2D pose estimation.

Again, some of the most successful approaches for human action recognition employ RNNs~\cite{Song_17, Du_15} to effectively model the temporal sequences of human skeleton data. Song \textit{et al.}~\cite{Song_17} proposed an end-to-end spatial and temporal attention model with LSTM for human action recognition from skeleton data. In~\cite{Du_15}, Du \textit{et al.} proposed an end-to-end hierarchical RNN network for skeleton-based action recognition. They divided the human skeleton into five different parts and then separately feed them into five sub-networks. 

Our approach to gait recognition is similar to these approaches. In this study, we have proposed a simple RNN architecture that effectively models the discriminative gait features in a temporal domain. 


