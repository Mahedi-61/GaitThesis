\chapter{Conclusion}\label{ch:conclusion}

\section{Summary of Our Work}
In this thesis, a novel feature extraction techniques were proposed from 2D human pose estimation to find the effective and discriminative gait features for robust gait recognition. We also present a novel RNN architecture which is much more simple, efficient and computationally inexpensive compared to the existing architectures proposed in literature. We considered human pose information as gait features for our network because it not only has rich gait representation capacity but also shows robustness toward the variation of carrying and clothing condition. Experimental results on challenging CASIA A and CASIA B gait dataset clearly depicts that the method proposed in this thesis outperforms the existing state-of-the-art methods in literature.


\section{Future Prospects of Our Work}
In future, we will employ more accurate pose estimation algorithm that can improve the recognition rate greatly especially in a large view variation. Thus, it will further boost our performance and lead us to achieve state-of-the-performance in cross-view gait recognition. Using a larger dataset containing thousands of subject will help us to develop a more stable network suitable for practical applications like real-time surveillance. 







