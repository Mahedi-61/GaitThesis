\chapter{Conclusion}\label{conclusion}

\section{Summary of Our Work}
In this paper, a novel features extraction technique was proposed from 2D human pose estimation to find the effective gait features for view-invariant gait recognition robust to covariate factors. We also present a novel RNN architecture which is much more simple, efficient and computationally inexpensive compared to other state-of-the-art architectures present in literature. We considered human pose information as gait features for our network because it not only has rich gait representation capacity but also shows robustness towards the variation of carrying and clothing condition. Experimental results on challenging CASIA A and CASIA B gait dataset clearly depicts that the method proposed in this paper outperforms the existing state-of-the-art methods in literature.


\section{Future Prospects of Our Work}
In future, more accurate pose estimation algorithm can improve cross-view recognition rate greatly especially in a large view variation, which will further boost our performance and lead us to achieve state-of-the-performance. Using a larger dataset containing thousands of subject will help us to develop a more stable network suitable for practical applications like real-time surveillance. 
\endinput