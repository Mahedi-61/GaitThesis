\chapter{Introduction}\label{intro}
Biometrics refers to the automatic identification or authentication of people by analyzing their physiological and behavioral characteristics. Physiological biometrics is related to the shape of body parts like fingerprints, shape of the hand, eye (iris and retina), etc., and is now used as the most stable means for authenticating and identifying people in reliable way. But, for efficient and accurate authentication, these traits require cooperation from the subject along with a comprehensive controlled environmental setup. Hence, these traits are not useful in surveillance systems. Behavioral biometrics such as signatures, gestures, gait, voice, etc., is related to a person\rq s behavior. But, these traits are more prone to changes depending on factors such as aging, injuries, or even mood. 

Gait recognition is a behavioral biometric modality that identifies a person based on the walking posture of that person. A unique advantage of gait as a biometric is that it offers recognition at a distance and at low-resolution without any user cooperation. That is why gait biometric signature is now considered the only likely identification method suitable for access control, covert video surveillance and forensic analysis which is not prone to spoofing attacks and signature forgery.

Due to the advantages of gait recognition, the past two decades have witnessed significant improvements of gait recognition system. However, unfortunately, there still exist many challenges that need to be addressed for robust gait recognition. The presence of various covariate factors like viewing angle, clothing, carrying heavy bags, wearing different shoes, variations in walking surface, and occlusions, etc., can badly affect people gait representation and drastically reduce the performances of gait recognition. Traditional body appearance-based methods in gait recognition are sensitive to these covariate factors. Because, the extraction of human silhouettes is affected by lighting changes. Even when the extraction step is performed correctly, the shape of the body depends significantly on covariate factors. Therefore, appearance-based methods are not completely robust towards these covariates change.

In this work, we consider human 2D pose sequence as our effective gait descriptors because it does not depend on people body appearance and shape. Therefore, it will be less affected by the variation of covariate factors. Again, in deep learning, recurrent neural networks (RNNs) have achieved promising performance in many sequence labeling tasks. The rationale behind their effectiveness for sequence-based tasks is their ability to capture long-range dependencies in a temporal context from sequence. So, the key to our proposed method is to develop a pose-based deep recurrent neural network to recognize human gait by extracting and modeling temporal dynamics of 2D pose sequence data.

For multi-view gait recognition, we also propose a two-stage network in which we first determine the walking direction, i.e. the viewpoint angle of the camera using a 3D convolutional network and later identify the subject using proposed RNN based temporal network trained on that particular angle. Our proposed two-staged network is far simpler and efficient in terms of time and space while outperforming present state-of-the-art networks on multi-view gait recognition.


The main contributions of this paper are summarized as follows:
\begin{itemize}
	\item We propose a novel RNN network with GRU architecture and devise several strategies to effectively extract and model the temporal dynamics of 2D pose sequence data for robust gait recognition. This work will open a new avenue open for gait recognition which no longer needs to calculate gait energy image (GEI) or expensive optical flows as gait descriptors.
	
	\item We also propose a two-stage network for multi-view gait recognition in which we first identify the walking direction by extracting spatio-temporal features using 3D convolution and then performs subject recognition using temporal network trained on that particular angle.
	
	\item The proposed pose-based RNN network achieves the best results on two challenging benchmark dataset CASIA A and CASIA B by outperforming other prevailing methods in single-view as well as multi-view gait recognition at a significant margin.
	
\end{itemize}


\section{What is Gait Recognition}
\section{Importance of Gait Recognition}
\section{Challenges Gait Recognition}
